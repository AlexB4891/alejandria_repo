% Options for packages loaded elsewhere
\PassOptionsToPackage{unicode}{hyperref}
\PassOptionsToPackage{hyphens}{url}
%
\documentclass[
]{article}
\usepackage{amsmath,amssymb}
\usepackage{lmodern}
\usepackage{iftex}
\ifPDFTeX
  \usepackage[T1]{fontenc}
  \usepackage[utf8]{inputenc}
  \usepackage{textcomp} % provide euro and other symbols
\else % if luatex or xetex
  \usepackage{unicode-math}
  \defaultfontfeatures{Scale=MatchLowercase}
  \defaultfontfeatures[\rmfamily]{Ligatures=TeX,Scale=1}
\fi
% Use upquote if available, for straight quotes in verbatim environments
\IfFileExists{upquote.sty}{\usepackage{upquote}}{}
\IfFileExists{microtype.sty}{% use microtype if available
  \usepackage[]{microtype}
  \UseMicrotypeSet[protrusion]{basicmath} % disable protrusion for tt fonts
}{}
\makeatletter
\@ifundefined{KOMAClassName}{% if non-KOMA class
  \IfFileExists{parskip.sty}{%
    \usepackage{parskip}
  }{% else
    \setlength{\parindent}{0pt}
    \setlength{\parskip}{6pt plus 2pt minus 1pt}}
}{% if KOMA class
  \KOMAoptions{parskip=half}}
\makeatother
\usepackage{xcolor}
\IfFileExists{xurl.sty}{\usepackage{xurl}}{} % add URL line breaks if available
\IfFileExists{bookmark.sty}{\usepackage{bookmark}}{\usepackage{hyperref}}
\hypersetup{
  pdftitle={Evaluación de Impacto de nuevos centros de salud sobre la tasa de movilidad para mujeres en edad reproductiva.},
  pdfauthor={Bajaña Alex; Chanatasig Evelyn; Heredia Aracely; Lombeida Esteban},
  hidelinks,
  pdfcreator={LaTeX via pandoc}}
\urlstyle{same} % disable monospaced font for URLs
\usepackage[margin=1in]{geometry}
\usepackage{graphicx}
\makeatletter
\def\maxwidth{\ifdim\Gin@nat@width>\linewidth\linewidth\else\Gin@nat@width\fi}
\def\maxheight{\ifdim\Gin@nat@height>\textheight\textheight\else\Gin@nat@height\fi}
\makeatother
% Scale images if necessary, so that they will not overflow the page
% margins by default, and it is still possible to overwrite the defaults
% using explicit options in \includegraphics[width, height, ...]{}
\setkeys{Gin}{width=\maxwidth,height=\maxheight,keepaspectratio}
% Set default figure placement to htbp
\makeatletter
\def\fps@figure{htbp}
\makeatother
\setlength{\emergencystretch}{3em} % prevent overfull lines
\providecommand{\tightlist}{%
  \setlength{\itemsep}{0pt}\setlength{\parskip}{0pt}}
\setcounter{secnumdepth}{5}
\ifLuaTeX
  \usepackage{selnolig}  % disable illegal ligatures
\fi

\title{Evaluación de Impacto de nuevos centros de salud sobre la tasa de
movilidad para mujeres en edad reproductiva.}
\usepackage{etoolbox}
\makeatletter
\providecommand{\subtitle}[1]{% add subtitle to \maketitle
  \apptocmd{\@title}{\par {\large #1 \par}}{}{}
}
\makeatother
\subtitle{Tópico: Datos y enfoque en investigaciones inclusivas}
\author{Bajaña Alex \and Chanatasig Evelyn \and Heredia
Aracely \and Lombeida Esteban}
\date{2022-05-29}

\begin{document}
\maketitle

{
\setcounter{tocdepth}{1}
\tableofcontents
}
\hypertarget{resumen}{%
\section{Resumen}\label{resumen}}

La provisión autónoma de servicios de salud se entiende como la
capacidad de prevenir, atender y curar las afecciones de la población de
una parroquia con los medios disponibles en la misma. De modo que las
parroquias que cuentan con suficientes recursos para ofrecer servicios
de salud a su población son consideradas autónomas. Las parroquias que
reciben más pacientes que sus residentes habituales son importadoras
netas, y aquellas que no pueden solventar las necesidades de servicios
de salud de su población son exportadoras netas.

La disposición geográfica de hospitales o centros de salud, de acuerdo a
la planificación gubernamental, en caso de ser bien ejecutada,
implicaría que los servicios de salud sean provistos de manera local,
considerando la ubicación geográfica, aspectos culturales y demografía
de su población.

La atención materna, ginecológica y neonatal es de vital importancia
para el bienestar de la población de cada parroquia. Por ende la
creación de centros de atención con estos servicios ayudarían a reducir
notablemente la tasa de movilidad de mujeres en edad reproductiva (10-49
años).

Con fuente en los registros administrativos del Ministerio de Salud
Pública: Recursos y Actividades de Salud y Egresos hospitalarios, es
posible encontrar cuántas mujeres se han atendido fuera de su parroquia
de residencia habitual, permitiendo indagar en el impacto que tienen los
centros de atención sobre la tasa de movilidad relacionada a
complicaciones en atención ginecológica, materna y neonatal.

Los autores pretenden utilizar regresiones logísticas o modelos
generalizados para determinar si la inclusión de nuevos centros de salud
afecta la probabilidad de que una parroquia sea considerada autónoma,
importadora o exportadora de pacientes. La interpretación de estos
resultados permitiría entender las necesidades por ser atendidas para
una mejor planificación y asignación de recursos a estas especialidades.

\end{document}
